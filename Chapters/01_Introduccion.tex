% Introducción

\chapter{Introducción} % Chapter title

\label{ch:introduccion} % For referencing the chapter elsewhere, use \autoref{ch:introduccion} 

%----------------------------------------------------------------------------------------
\graffito{Cloud Computing: Son el conjunto de servicios y recursos que funcionan sobre Internet para desplegar aplicaciones.\bigskip}
\graffito{On-premise: En esta configuración el cliente se encarga de hospedar sus aplicaciones en infraestructura propia (servidores, conexiones y demás).\bigskip}
\graffito{LibCloud: es una librería para Python para la interacción con los servicios cloud mas populares usando una \acs{API} unificada.\bigskip}
\graffito{Docker: Es una tecnología que permite deplegar aplicaciones dentro de contenedores para facilitar su automatización.\bigskip}

El \textit{Cloud Computing} ha facilitado la puesta en marcha de todo tipo de aplicaciones sin importar la necesidad de recursos que esta tenga. Esto es un punto importante en las tecnologías del cloud, permitirle al cliente iniciar con una  infraestructura o servicio de base que luego puede ser escalada a la medida de las necesidades que la aplicación requiera. En contraposición de una infraestructura \textit{On-premise} las tecnologías cloud se adaptan facilmente a los nuevos requerimientos de la aplicación.\bigskip

\appName permitirá usar varios de los Drivers que ofrece \textit{LibCloud}\footnote{\url{https://libcloud.apache.org/about.html}} para interactuar con \acs{aws} sobre una interfaz web, En esta interfaz el usuario podrá hacer hacer una comparación entre los costos de aprovisionar su configuración de recursos sobre varios provedores y adionalmente podrá ser asistido sobre el despliegue de estos recursos sobre \acs{aws}.\bigskip







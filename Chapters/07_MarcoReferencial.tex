% Planteamiento del Problema

\chapter{Marco referencial} % Chapter title

\label{ch:marcoReferencial} % For referencing the chapter elsewhere, use \autoref{ch:introduction} 
\section{Glosario}

\refstepcounter{dummy}
%\addcontentsline{toc}{chapter}{Glosario} % Uncomment if you would like the acronyms to appear in the table of contents
\pdfbookmark[1]{Glosario}{Glosario} % Bookmark name visible in a PDF viewer
\markboth{\spacedlowsmallcaps{Glosario}}{\spacedlowsmallcaps{Glosario}}



\begin{acronym}[API]
    \acro{API}{Es una Interfaz de Programación de Aplicaciones, a veces tambien conocido como librería de código.}
\end{acronym}  

\begin{acronym}[Endpoints]
    \acro{Endpoints}{Comunmente las rutas en un servicio web que responden a las peticiones que le son solicitadas.}
\end{acronym}  

\begin{acronym}[Let'sEncrypt]
    \acro{Let'sEncrypt}{Es una entidad certificadora gratuita de lucro para expedición de cetificados SSL/TLS.}
\end{acronym}  

\begin{acronym}[React]
    \acro{React}{Es una librería de Javascript para la construcción de aplicaciones web reactivas.}
\end{acronym}  

\begin{acronym}[HTTPS]
    \acro{HTTPS}{El protocolo que usa la web para enviar y recibir tráfico  encriptado a través de una capa de sockets seguros.}
\end{acronym}  

\begin{acronym}[AES]
    \acro{AES}{Es el estandar de encriptación avanzado para el cifrado de datos.}
\end{acronym}  

\begin{acronym}[Bots]
    \acro{Bots}{Son algoritmos que realizan tareas periódicas en Internet y la web generalmente para la recopilación de datos. }
\end{acronym}  

\begin{acronym}[API REST]
    \acro{API REST}{Es una interfaz de programación de aplicaciones que permite la comunicación a través de la web usando el protocolo HTTP.}
\end{acronym}  

\begin{acronym}[LibCloud]
    \acro{LibCloud}{Es la librería de Python mas completa hasta el momento para interactuar con los servicios se proveedores de Cloud Computing.}
\end{acronym}

\begin{acronym}[aws]
    \acro{aws}{Amazon Web Services, es la empresa lider en servicios de Cloud Computing.}
\end{acronym}  

\begin{acronym}[Azure]
    \acro{Azure}{Es una empresa de Microsoft que ofrece servicios de Cloud Computing}
\end{acronym}

\begin{acronym}[Google Cloud]
    \acro{Google Cloud}{Es una empresa de Cloud Computing fundada por Google en el 2008.}
\end{acronym}

\begin{acronym}[IBM Cloud]
    \acro{IBM Cloud}{Es el servicio de Cloud Computing prestado por la empresa IBM.}
\end{acronym}

\begin{acronym}[DigitalOcean]
    \acro{DigitalOcean}{Es un proveedor de servicios de Cloud Computing que proviene de New York fundada en el año 2011.}
\end{acronym}

\begin{acronym}[HostDime]
    \acro{HostDime}{Esta empresa tiene presencia de infraestructura de servicios de Cloud Computing en Colombia.}
\end{acronym}

\begin{acronym}[HostingRED]
    \acro{HostingRED}{Es una empresa de Cloud Computing con centros de datos ubicados en Bogotá Colombia además de Estados Unidos y Canadá.}
\end{acronym}
             
\begin{acronym}[UI]
    \acro{UI}{Referente al diseño que tiene la interfaz de un proyecto de software o tecnología.}
\end{acronym}

\begin{acronym}[UX]
    \acro{UX}{Referente al diseño de la experiencia del usuario final en una aplicación de software o tecnología.}
\end{acronym}

\begin{acronym}[XPath]
    \acro{XPath}{Es un lenguaje de programación para formular patrones que permiten encontrar coincidencias en documentos de marcado de etiquetas.}
\end{acronym}

\begin{acronym}[Docker-compose]
    \acro{Docker-compose}{Es una tecnología para la automatización y orquestación de aplicaciones a partir de contenedores de Docker.}
\end{acronym}
%----------------------------------------------------------------------------------------
\section{Estado del arte}


\section{Marco teórico}
% Planteamiento del Problema

\chapter{Marco referencial} % Chapter title

\label{ch:marcoReferencial} % For referencing the chapter elsewhere, use \autoref{ch:introduction} 



%\refstepcounter{dummy}
%\addcontentsline{toc}{chapter}{Glosario} % Uncomment if you would like the acronyms to appear in the table of contents

%----------------------------------------------------------------------------------------
\section{Estado del arte}
Este documento toma como referencia dos ejes temáticos principalmente, los modelos de cobro de los servicios de \acrshort{CC} y las tecnologías referentes a la administración, despliegue y orquestación de recursos de \acrshort{CC} bajo un esquema de iteroperabilidad.\bigskip

Los modelos de precios definidos son usados

\section{Marco teórico}
En esta sección se hace una corta descripción de los servicios ofrecidos por los proveedores \acrshort{CC}
\chapter{Alcances de la propuesta} % Chapter title

\label{ch:alcances} % For referencing the chapter elsewhere, use \autoref{ch:alcances} 

%----------------------------------------------------------------------------------------

\graffito{El archivo \texttt{robots.txt} ubicado en la raiz de un sitio web indica a los \acs{Bots} que información es de caracter privado o no debe ser indexada.\bigskip}

\appName pretende recuperar de la Web información de los principales proveedores de Cloud entre los que están \acs{aws}, \acs{Azure}, \acs{Google Cloud}, \acs{IBM Cloud} y \acs{DigitalOcean}. \appName también tiene en cuenta proveedores locales (en Colombia) como \acs{HostDime}, \acs{HostingRED}; Se espera que el registro de nuevos proveedores se haga por cualquier usuario y se valide por el administrador de la aplicación.\bigskip

\appName tiene soporte básico de admistración para los recursos de tipo \textit{Compute}, \textit{Storage} y \textit{Container} solamente para el proveedor \acs{aws} pero con la posibilidad de extender el proyecto a nuevos tipos de servicios teniendo en cuenta las disponibilidad de controladores\footnote{\url{https://libcloud.readthedocs.io/en/stable/supported_providers.html}} de la API de \textit{LibCloud}. \bigskip

La implementación de la interfaz web se construye en \acs{React} a partir de la librería \textit{CoreUI}\footnote{\url{https://github.com/coreui/coreui-react/}}, es la razón por la que esta propuesta no considera el diseño de interfaz de usuario \acs{UI} o experiencia de usuario \acs{UX} como un componente principal.\bigskip

La información que se pretende recuperar de la Web de los proveedores de Cloud tiene en cuenta las limitaciones técnicas y legales que los mismos proveedores incluyen en sus politicas y en el archivo \texttt{robots.txt} para evitar incurrir en problemas legales. \bigskip

La implementación de la aplicación con todos sus componentes tiene como objetivo mostrar la aplcación en funcionamiento, sin embargo los aspectos técnicos adicionales referentes a su despliegue como pruebas de rendimiento, pentesting y mantenimiento no se tienen en cuenta en esta propuesta.
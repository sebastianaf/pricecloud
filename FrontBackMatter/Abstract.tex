% Abstract

%\renewcommand{\abstractname}{Abstract} % Uncomment to change the name of the abstract

\pdfbookmark[1]{Abstract}{Resumen} % Bookmark name visible in a PDF viewer


\begingroup
\let\clearpage\relax
\let\cleardoublepage\relax
\let\cleardoublepage\relax

\chapter*{Resumen}
El aprovisionamiento de recursos en provedores de servicios web es un proceso que cada vez toma mas relevancia en aplicaciones que han apostado por la tendencia a migrarse al cloud computing. La viabilidad de un proyecto de software esta comprometida con los costos que demanda usar los recursos de estos provedores, gracias a los modelos comerciales actuales es común el "pago por uso" \textit{Pay as You Go} donde se discretiza el cobro en términos de peticiones, duración, espacio, cantidad de usuarios, y demás variables. \\

La elección de los servicios necesarios para aprovisionar un proyecto de software es sin duda un tema complejo cuando se incluyen todas las variables en cuestión y ademas cuando se consideran los diferentes recursos de estos proveedores. \\

La aplicación web propuesta en este trabajo ofrece la posibilidad de integrar la fase de evaluación de costos junto con el aprovisionamiento de recursos en una misma interfaz con fin de agilizar y facilitar la elección de los servicios que minimicen los costos de despligue para un proyecto de software con respecto a sus necesidades iniciales.

\endgroup			

\vfill
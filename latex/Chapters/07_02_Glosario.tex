\newglossaryentry{Endpoint}{
    name={Endpoint},
    description={Comunmente las rutas en un servicio web que responden a las peticiones que le son solicitadas}
}

\newglossaryentry{Let's Encrypt}{
    name={Let's Encrypt},
    description={Es una entidad certificadora gratuita de lucro para expedición de cetificados SSL/TLS}
}

\newglossaryentry{React}{
    name={React},
    description={Es una biblioteca de Javascript para la construcción de aplicaciones web reactivas}
}

\newglossaryentry{Bot}{
    name={Bot},
    description={Es un programa de computador o script que realiza tareas periódicas de recopilación de datos generalmente sobre Internet y la web}
}

\newglossaryentry{LibCloud}{
    name={LibCloud},
    description={Es una de las bibliotecas escrita en lenguaje Python mas completa hasta el momento para interactuar con los servicios de los \acrshortpl{CCSP}}
}

\newglossaryentry{Microsoft Azure}{
    name={Microsoft Azure},
    description={Es una plataforma de computación en la nube que ofrece servicios de \acrshort{CC}}
}

\newglossaryentry{Google Cloud}{
    name={Google Cloud},
    description={Es una empresa de \acrshort{CC} fundada por Google en el 2008}
}

\newglossaryentry{IBM Cloud}{
    name={IBM Cloud},
    description={Es el servicio de \acrshort{CC} prestado por la empresa IBM}
}

\newglossaryentry{Digital Ocean}{
    name={Digital Ocean},
    description={Es un proveedor de servicios de \acrshort{CC} que proviene de New York fundada en el año 2011}
}

\newglossaryentry{Host Dime}{
    name={Host Dime},
    description={Esta empresa tiene presencia de infraestructura de servicios de \acrshort{CC} en Colombia}
}

\newglossaryentry{Hosting RED}{
    name={Hosting RED},
    description={Es una empresa de \acrshort{CC} con centros de datos ubicados en Bogotá Colombia además de Estados Unidos y Canadá}
}

\newglossaryentry{XPath}{
    name={XPath},
    description={Es un lenguaje de programación para formular patrones que permiten encontrar coincidencias en documentos de marcado de etiquetas}
}

\newglossaryentry{Docker Compose}{
    name={Docker Compose},
    description={Es una tecnología para la automatización y orquestación de aplicaciones a partir de contenedores de \gls{Docker}}
}

\newglossaryentry{User Interface}{
    name={User Interface},
    description={Referente al diseño que tienen las interfaces de un proyecto de software o tecnología}
}

\newglossaryentry{Amazon Web Services}{
    name={Amazon Web Services},
    description={Es una empresa lider mundial en servicios de \acrshort{CC}}
}

\newglossaryentry{User Experience}{
    name={User Experience},
    description={Referente al diseño de la experiencia del usuario final en una aplicación de software o tecnología}
}

\newglossaryentry{Cloud Computing}{
    name={Cloud Computing},
    description={El CC es una tendencia de servicios y recursos que funcionan sobre Internet para facilitar el despliegue, uso y disponibilidad de aplicaciones de Software}
}

\newglossaryentry{Representational State Transfer API}{
    name={Representational State Transfer API},
    description={Es una interfaz de programación de aplicaciones que permite la comunicación a través de la web usando el protocolo HTTP}
}

\newglossaryentry{Hypertext Media Transfer Protocol Secure}{
    name={Hypertext Media Transfer Protocol Secure},
    description={El protocolo que usa la web para enviar y recibir tráfico  encriptado a través de una capa de sockets seguros}
}

\newglossaryentry{Cloud Computing Provider}{
    name={Cloud Computing Provider},
    description={Empresa que presta servicios de Computación en la nube}
}

\newglossaryentry{CoreUI}{
    name={CoreUI},
    description={Es una biblioteca de React que ofrece una versión de uso libre de componentes gráficos escritos en Javascript y TypeScript y apoyada en en los estilos de \gls{Bootstrap}}
}

\newglossaryentry{Bootstrap}{
    name={Bootstrap},
    description={Es una biblioteca de estilos para aplicaciones web definidos principalmente en lenguaje SCSS}
}

\newglossaryentry{Docker Hub}{
    name={Docker Hub},
    description={Es el nombre del sitio web que contiene las librerías de imágenes de contenedores de \gls{Docker}}
}

\newglossaryentry{Docker}{
    name={Docker},
    description={Es una plataforma de contenerización para la implementación microservicios, abstrae recursos como redes, volumenes de almacenamiento, y nodos de cómputo}
}

\newglossaryentry{Framework}{
    name={Framework},
    description={Conjunto de tecnologías que trabajan juntas para darle desarrollo a un proyecto que tiene objectivos específicos}
}
\newglossaryentry{Plug-in}{
    name={Plug-in},
    description={Son complementos a las funcionalidades de base que tiene una aplicación de software, pueden ser incorporados por el usuario a discreción}
}

\newglossaryentry{Hosting}{
    name={Hosting},
    description={Es un servicio relacionado generalmente con el alojamiento de un sitio web en una empresa que presta el servicio}
}

\newglossaryentry{Infracost}{
    name={Infracost},
    description={Es un proyecto de estimación de costos para Terraform y Kubernetes que ofrece una \acrshort{API} pública para recuperar información de costos de los principales proveedores de servicios de \acrshort{CC}}
}

\newglossaryentry{GraphQL}{
    name={GraphQL},
    description={Es un lenguaje de consulta para \acrshortpl{API} que permite recuperar información de manera eficiente y flexible}
}

\newglossaryentry{App Engine}{
    name={App Engine},
    description={Es un \acrshort{PaaS} de \acrshort{GCP} que permite desplegar aplicaciones web y servicios de backend y frontend}
}

\newglossaryentry{NextJS}{
    name={NextJS},
    description={Es un framework de React usado por grandes empresas para el desarrollo de aplicaciones web tipo \acrshort{SSR}}
}

\newglossaryentry{NestJS}{
    name={NestJS},
    description={Es un framework de NodeJS para el desarrollo de aplicaciones web del lado de backend, se caracteriza por su modularidad y escalabilidad.}
}

\newglossaryentry{Material UI}{
    name={Material UI},
    description={Es una librería de componentes que implementa el sistema de diseño de Material Design de \emph{Google}}
}

\newglossaryentry{Nginx Proxy Manager}{
    name={Nginx Proxy Manager},
    description={Es un proxy inverso administrado por una interfaz Web sobre NodeJS y con las ventajas de Nginx}
}

\newglossaryentry{PostgreSQL}{
    name={PostgreSQL},
    description={Es un sistema de gestión de bases de datos relacionales de código abierto y gratuito conocido por su confiabilidad y robustez, este simplifica algunas labores de consultas y compatibilidad entre servidores de bases de datos}
}

\newglossaryentry{TypeORM}{
    name={TypeORM},
    description={Es un ORM agnóstico del proveedor para NodeJS que permite la conexión a bases de datos relacionales}
}

\newglossaryentry{Sequelize}{
    name={Sequelize},
    description={Definido como un \acrfull{ORM} permite definir el modelo de la base de datos usando el lenguaje del backend }.
}
\chapter{Objetivos} % Chapter title

\label{ch:objetivos} % For referencing the chapter elsewhere, use \autoref{ch:introduction} 

%----------------------------------------------------------------------------------------

\graffito{El lenguaje \gls{XPath} es un recurso que permite describir patrones para hallar coincidencias en los \glspl{Endpoint} de un sitio Web. \bigskip}

\graffito{La recuperación de información en la web inicia con el Crawling que permite obtener las páginas web de objetivo para luego aplicar Web Scraping que recopila información de interés.\bigskip}

\section{Objetivo general}
Diseñar, e implementar un prototipo de aplicación web para informar las tarifas de aprovisionamiento de recursos tipo \emph{Compute}, \emph{Storage} y \emph{Container} en el \acrshort{CC} y administrar su despliegue.\bigskip

\section{Objetivos específicos}
\begin{enumerate}
    \item Definir un modelo general de evaluación de costos para el análisis de los servicios en el \acrshort{CC} usando información recuperable y relevante de la Web de sus proveedores.
    
    \item Diseñar e implementar un prototipo basado en microservicios que recopile periodicamente y almacene las tarifas de los principales \acrshortpl{CCSP}.
    
    \item Diseñar e implementar un prototipo de aplicación web basada en microservicios para acceder a la información de costos de los proveedores y administrar los recursos de un usuario usando tecnologías agnósticas del proveedor.
    
\end{enumerate}
% Introducción
\chapter{Introducción} % Chapter title

\label{ch:introduccion} % For referencing the chapter elsewhere, use \autoref{ch:introduccion} 

%----------------------------------------------------------------------------------------
\graffito{El \acrshort{CC} es una tendencia de servicios y recursos que funcionan sobre Internet para facilitar el despliegue, uso y disponibilidad de aplicaciones de Software.\bigskip}
\graffito{En una configuración On-premise el cliente se encarga de hospedar sus aplicaciones en uno o varios servidores en su propia infraestructura.\bigskip}
\graffito{El modelo comercial Pay as You Go permite al cliente pagar solo por lo que ha usado después de haberlo usado.\bigskip}


El \acrlong{CC} ha facilitado el despliegue de todo tipo de aplicaciones con un uso adaptable de recursos, esto es un punto importante del cloud, permitirle al cliente iniciar con recursos de base que luego pueden ser escalados a la medida de las necesidades que la aplicación requiera. En contraposición de una infraestructura \emph{On-premise} las tecnologías cloud se adaptan en tiempo real a los requerimientos cambiantes de la demanda de estas aplicaciones y le permiten al cliente invertir recursos vitales en procesos mas relevantes eliminando los costos de mantenimiento e infraestructura tecnológica.\bigskip

El aprovisionamiento de recursos en proveedores de servicios cloud es un proceso que cada vez toma mas relevancia en aplicaciones que han apostado por la tendencia a migrarse al \acrshort{CC}. Sin embargo a la hora de elegir estos recursos se deben tener en cuenta aspectos como la arquitectura del proyecto, el tipo de recursos, su ubicación geográfica, las políticas donde residirán, la reputación del proveedor y por supuesto de su costo.\bigskip

Uno de los aspectos clave para ayudar a garantizar la continuidad de un proyecto de software son precisamente sus costos fijos; el cloud los discretiza con el modelo \emph{Pay as You Go} en términos de peticiones, duración, poder de cómputo, espacio de almacenamiento, usuarios, y demás variables que le agregan aún mas complejidad a la decisión de escoger el proveedor adecuado para los recursos requeridos por el proyecto. \bigskip

\appName permite interactuar sobre una interfaz web donde el usuario puede tomar una decisión informada a partir del conocimiento de los costos de su configuración de recursos sobre varios \acrshort{CCSP} y de manera complementaria es asistido en su aprovisionamiento sobre \acrshort{AWS} .\bigskip

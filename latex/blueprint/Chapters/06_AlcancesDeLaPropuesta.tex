\chapter{Alcances de la propuesta} % Chapter title

\label{ch:alcances} % For referencing the chapter elsewhere, use \autoref{ch:alcances} 

%----------------------------------------------------------------------------------------

\graffito{El archivo \texttt{robots.txt} ubicado en la raiz de un sitio web indica a los \glspl{Bot} que información es de caracter privado o no debe ser indexada.\bigskip}

\appName recupera información de precios los principales \acrshortpl{CCSP}, estos son \acrshort{AWS}, \gls{Microsoft Azure}, \gls{Google Cloud}.\bigskip

\appName tiene soporte básico de administración para los recursos de tipo \emph{Compute} y \emph{Storage}, a la fecha de publicación de este documento el controlador para operaciones de \emph{Container} se encuentra en etapa experimental y su implementación no fue exitosa. Las dos operaciones listadas se aplicaron para el proveedor \acrshort{AWS}, el proyecto de \emph{Python} esta construido buenas prácticas de desarrollo y  los principios de la programación modular,es proyecto tiene la posibilidad de extender la implementación de mas controladores en el futuro teniendo en cuenta la disponibilidad de  de la librería\footnote{\url{https://libcloud.readthedocs.io/en/stable/supported_providers.html}} de \gls{LibCloud}. \bigskip

La implementación de la interfaz web se construyó en el \emph{framework} \gls{NextJS} con la librería de \gls{React} y \gls{Material UI}\footnote{\url{https://mui.com/}}, el objetivo principal es servir como una fuente de información mas que proveer una visualización atractiva, el desarrollo estuvo soportado por la plantilla \emph{BloomUI}\footnote{\url{https://bloomui.com/}} y no tuvo un componente complementario de \acrshort{UI} y \acrshort{UX}.\bigskip

La información recopilada es de uso libre y proporcionada por \emph{Infracost} a través de su \emph{Cloud Pricing API}\footnote{\url{https://www.infracost.io/docs/cloud_pricing_api/overview/}}. La información de precios es actualizada semanalmente y se almacena en una base de datos \emph{PostgreSQL}.\bigskip

La implementación de la aplicación con todos sus componentes no tiene aspectos técnicos adicionales referentes a su despliegue como pruebas de rendimiento, pentesting y mantenimiento.
\chapter{Resultados del trabajo} % Chapter title

\label{ch:resultados} % For referencing the chapter elsewhere, use \autoref{ch:introduction} 

\graffito{Todos los servicios de la aplicación están disponibles y listos para desplegar con un \emph{script} de \emph{docker-compose} publicados en el repositorio oficial del proyecto. el \emph{script} se encarga de desplegar cada parte en imágenes de \gls{Docker} con \emph{Dockerfile}s y sus instrucciones para el proceso de \emph{building}.\bigskip}

\graffito{La comunicación pública de todos los web services esta encriptada usando el protocolo \acrshort{HTTPS} con certificados SSL/TLS de \gls{Let's Encrypt}, los datos sensibles almacenados usan el algoritmo de cifrado simétrico de dos vías \acrshort{AES}. \bigskip}

El proyecto cumple los objetivos específicos mencionados anteriormente como se describe a continuación.


\begin{enumerate}
  \item El primer resultado con respecto al primer objetivo específico es la desarrollo e implementación de un modelo comparativo de costos que se ejecuta sobre la aplicación web. Este modelo es un punto de referencia que mide los servicios agrupados por categoría, costo y ubicación geográfica.

  \item El segundo resultado con respecto al segundo objetivo específico es el desarrollo e implementación de dos microservicios encargados de administrar la información relevante de los precios de los \acrshort{CCSP} y la lógica general de la aplicación. los microservicio se encuentran actualmente en línea, funcionan sin interrupciones 24/7 y su documentación está disponible en \url{https://api.dev.pricecloud.org/docs}.

  \item El tercer resultado con respecto al tercer objetivo específico es un prototipo de aplicación web que está desplegada de manera pública en \url{https://pricecloud.org}, es una interfaz Web que puede ser desplegada por cualquier otro usuario sobre su propia infraestructura, su código fuente al igual que los microservicios mencionados están disponibles en el repositorio del proyecto \url{https://github.com/sebastianaf/pricecloud}.

\end{enumerate}



%----------------------------------------------------------------------------------------
